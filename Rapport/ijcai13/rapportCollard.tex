%%%% ijcai11.tex

\typeout{IJCAI-13 Instructions for Authors}

% These are the instructions for authors for IJCAI-13.
% They are the same as the ones for IJCAI-11 with superficical wording
%   changes only.

\documentclass{article}
% The file ijcai13.sty is the style file for IJCAI-13 (same as ijcai07.sty).
\usepackage{ijcai13}
\usepackage{graphicx}
\usepackage{algorithmic}
\usepackage{algorithm}
% Adaptation au français
\usepackage[utf8]{inputenc}

% Use the postscript times font!
\usepackage{times}
% Package pour faire une flèche
\usepackage{textcomp}


\title{Rapport de projet}
\author{Théo Q, Corto C, Brandon FL \\
Université de Nice\\
France}

\begin{document}

\maketitle


\begin{abstract}
Dans ce projet nous étudions l'impact de la proportion de fourmis chercheuse et ramasseuses dans une fourmilière sur la récolte de nourriture. Nous utilisons pour cela un algorithme de type "Ant colony optimization" en Netlogo. Nos fourmis recherchent des tas de nourritures plus ou moins éloignés et utilisent des phéromones pour les localiser. Nous avons obtenus des données expérimentales en faisant varier graduellement la répartition des fourmis au sein de la fourmilières. Ces données ont ensuite étés  étudiés avec le logiciel R pour extraire un modèle mathématique. L'analyse de la fonction résultats, tempsDecouvertesTas(ProportionChercheuse) donne la répartition optimale :
\end{abstract}
\section{Introduction}
\subsection{Contexte}
Dans le cadre de l'UE "Projet Scientifique Informatique" 
$\bullet$ Apprentissage d’outils professionnels de pr´eparation de
documents
\subsection{Etat de l'art}
Les "Algorithmes de colonies de fourmis", "ACO" en anglais, forment une catégorie d'algorithmes d'optimisation basés sur la modélisation d'une colonie de fourmis. En effet bien qu'ayant des capacités individuelles limités ces insectes arrivent collectivement à des solution optimales de problèmes complexes. 
L'ACO est aujourd’hui utilisé en particulier dans les domaines suivants : routage de réseau, transport urbain 
\subsection{Questions scientifiques}
\section{Modélisation}
\subsection{Hypothèses simplificatrices}
\subsection{Description du modèle}
\section{Simulation}
\subsection{Cadre expérimental}
\subsection{Protocole expérimental}
\section{Résultats}
\section{Discussion}
\section*{Conclusion et perspectives}

\end{document}

%%%% ijcai11.tex

\typeout{IJCAI-13 Instructions for Authors}

% These are the instructions for authors for IJCAI-13.
% They are the same as the ones for IJCAI-11 with superficical wording
%   changes only.

\documentclass{article}
% The file ijcai13.sty is the style file for IJCAI-13 (same as ijcai07.sty).
\usepackage{ijcai13}
\usepackage{graphicx}
\usepackage{algorithmic}
\usepackage{algorithm}
% Adaptation au français
\usepackage[utf8]{inputenc}

% Use the postscript times font!
\usepackage{times}
% Package pour faire une flèche
\usepackage{textcomp}


\title{Rapport de projet}
\author{Théo Q, Corto C, Brandon FL \\
Université de Nice\\
France}

\begin{document}

\maketitle


\begin{abstract}
Dans ce projet nous étudions l'impact de la proportion de fourmis chercheuse et ramasseuses dans une fourmilière sur la récolte de nourriture. Nous utilisons pour cela un algorithme de type "Ant colony optimization" en Netlogo. Nos fourmis recherchent des tas de nourritures plus ou moins éloignés et utilisent des phéromones pour les localiser. Nous avons obtenus des données expérimentales en faisant varier graduellement la répartition des fourmis au sein de la fourmilières. Ces données ont ensuite étés  étudiés avec le logiciel R pour extraire un modèle mathématique. L'analyse de la fonction résultats, tempsDecouvertesTas(ProportionChercheuse) donne la répartition optimale :  !-AJOUTER-!
\end{abstract}
\section{Introduction}
\subsection{Contexte}
Les "Algorithmes de colonies de fourmis", "ACO" en anglais, forment une catégorie d'algorithmes d'optimisation basés sur la modélisation d'une colonie de fourmis. En effet bien qu'ayant des capacités individuelles limités ces insectes arrivent collectivement à des solution optimales de problèmes complexes. 



Dans le cadre de l'UE "Projet Scientifique Informatique nous avons choisi le sujet "Colonie de fourmis" d'étude parmi de nombreux proposés. Le modèle de base contenait un seul type de fourmis et trois tas de nourritures fixes. Nous l'avons alors perfectionné.
\subsection{Etat de l'art}
L'ACO est de plus en plus utilisé aujourd'hui dans de très nombreux domaines. Plus particulièrement dans les domaines suivants : planification, routage réseau, affectation, ensembles, conception de circuit nanoélectroniques, traitement d'image etc. Tout les problèmes pouvant se ramener a la recherche d'un chemin optimal dans un graphe en général.
\subsection{Questions scientifiques}
\section{Modélisation}
\subsection{Hypothèses simplificatrices}
\subsection{Description du modèle}
\section{Simulation}
\subsection{Cadre expérimental}
\subsection{Protocole expérimental}
\section{Résultats}
\section{Discussion}
\section*{Conclusion et perspectives}

\end{document}

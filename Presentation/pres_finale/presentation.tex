\documentclass[11pt]{beamer}
\usetheme{Amsterdam}
\usepackage[utf8]{inputenc}
\usepackage{amsmath}
\usepackage{amsfonts}
\usepackage{amssymb}
\usepackage{hyperref}
\author{Callerisa \and Fontany-Legall \and Qui}
\title{Modélisation d'une colonie de fourmis}
%\setbeamercovered{transparent} 
%\setbeamertemplate{navigation symbols}{} 
%\logo{} 
\institute{Université de Nice} 
\date{\today} 
%\subject{} 
\begin{document}

\begin{frame}

\titlepage
\end{frame}

\begin{frame}
\tableofcontents
\end{frame}

\begin{frame}
\frametitle{Introduction}
\framesubtitle{Contexte}
\begin{block}{ACO}
\begin{itemize}
\item Ant colony optimisation
\item Travelling Salesman Problem
\item Exploration de graphes
\item Besoin d'étudier le comportement des fourmis
\end{itemize}
\end{block}
\end{frame}

\begin{frame}
\frametitle{Introduction}
\framesubtitle{Recherche de nourriture}
\begin{itemize}
\item Chercheuses et ramasseuses
\item Chercheuses partent en éclaireur
\item Reviennent quand elles trouvent une source de nourriture et laissent des phéromones
\item Ramasseuses suivent la trace jusqu'à la nourriture 
\end{itemize}
\end{frame}

\begin{frame}
\frametitle{Problématique}
\begin{block}{Direction des recherches}
\begin{itemize}
\item La présence de chercheuses a-t-elle une réelle influence sur l'efficacité de la récolte ?
\item Si oui, quelle est cette influence ?
\item Comment évolue-t-elle en fonction de le proportion de chercheuses ? 
\end{itemize}
\end{block}
\end{frame}

\begin{frame}
\frametitle{Modélisation}
\begin{block}{Hypothèses simplificatrices}
\begin{itemize}
\item Deux type de fourmis uniquement
\item Retrouver le nid à l'instinct
\item Pas de rivalité avec une autre colonie
\end{itemize}
\end{block}
\end{frame}

\end{document}
